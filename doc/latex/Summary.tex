%Summary and Outlook

To conclude our simulation study, our results support the general idea that network structure and qualities of the ideas held in them may mutually influence each other. The more rigid the structure of a network was, the more likely that the intra-idea distance and neighbourhood index of the network was larger. These two features (the intra-idea distance and neighbourhood index) varied more with the innovation rate $\alpha$ for less structured networks, and varied more with the complex contagion threhold $\delta$ for the most rigid structure - the caveman network. Parameter values ($\phi$, $\delta$, and $\alpha$) seemed to vary the average dominance time more than the network structures did, whereas the frequency of dominance of ideas did not decrease on average in the long run with any network structure.

A network in which the pattern of ideas held by the nodes are `in accord' within the clusters of the caveman network maintained more of the structure features of a caveman network. These values varied with values of rewiring probabilities $\phi$ and complex contagion threholds $\delta$. Networks with a random pattern of ideas or with a pattern with more `disaccord' within the clusters resulted in a structure that began to resemble a random graph more than a caveman network. These networks' values varied more with the parameter $\phi$ of rewiring. Nodes given the chance to rewire with nodes that share the same idea had more opportunity to do so in the random and `disaccord' idea patterns than in the pattern which already had much accord.

Thus, in addition to the structure of networks and the pattern of ideas in them, complex contagion thresholds, innovation rates, and the probability of creating new connections (while reflecting `preferences' of being connected with like-minded others) do influence features of the network. Several extensions to the proposed model could be investigated to better understand the relationships between network structure and idea distribution.

\paragraph{Rewiring criteria}
Future simulations may compare the emerging features of idea distributions not just between network structures, but also between different rewiring criteria. It is not clear how much of the features of the idea distribution in this simulation study was a result of the network structure or of the `preference' that nodes had in rewiring to like-minded nodes. Therefore, these results could be compared to (1) random rewiring or, to allow structure to play a larger role, (2) to allow random rewiring to nodes that are at most a distance of three nodes away. This is somewhat more realistic since most people connect with individuals who are somewhat in their vicinity through mutual connections. 

\paragraph{Complex contagion and innovation}
Considering complex contagion, novel ideas in our model were at a disadvantage for spreading in the network. Allowing novel ideas to have a larger influence weight may be one way to grant the ability of them to spread to other nodes. Alternatively, the complex contagion threshold for novel ideas may be lowered.

\paragraph{Random idea adoption}
Future simulations may investigate the effects of network structure on idea distributions by comparing to a `benchmark' model. This model would update the ideas of nodes at random, and thus the effects of connections may be better observed within network structures and then compared between network structures.



