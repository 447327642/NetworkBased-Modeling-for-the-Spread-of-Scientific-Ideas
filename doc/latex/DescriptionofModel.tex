%Description of the Model


The model used here is based on a study by \citet*{HN2006}.  Each simulation begins with a specified network structure as well as a distribution of the `idea' (or state) of the nodes. At each time step a node either changes its idea to that of one of its neighbors' ideas if its frequency surpasses a defined threshold, rewires to connect with a node that has the same idea, or generates a novel idea (this is the innovation parameter). 


Given the network structure and node states, three parameters are introduced: $\phi$ (probability of rewiring), $\alpha$ (probability of innovation), and $\delta$ (node threshold). As in \citet*{HN2006}, $\phi$ is a value from zero to one, and is the probability that one of the edges of a chosen node \emph{i} \textit{i} will be changed to connect to another node j that i is unconnected with. We decided to add one more criterion to this definition: node j is a node that has the same idea as node i. This encourages the simulations to reflect a common tendency of individuals to seek out others who think like them.


At each time step a node may `come up with a new idea' with a probability of $\alpha$. This value is small to reflect that novel ideas are not frequently observed.


We introduced a node threshold $\delta$ to the general model in order to investigate the behaviour of complex contagion as opposed to simple contagion. This was motivated by a study by \citet*{CM2007}. Simple contagion is well suited for modeling the spread of diseases since they may often be passed on by a single contact with an uninfected individual. However, as our intuition may suggest, and as studies have shown, the spread of other kinds of innovations require several exposures before they are adopted by individuals; these dynamics of diffusion refer to complex contagion.


\subsection{Networks}

For the purposes of our simulations, we used four network structures. These structures can be characterized by properties such as average shortest path lengths, clustering coefficients, and the degree of connectivity. For a brief definition of these properties (which will be used in the results section), please see Table ***** Below are short descriptions of each network structure.

\subsubsection{Random Graph}

A variant of the Erd\H{o}s-R\'enyi random graph model \citep*{ER1960}, and implemented by \citet*{BS2011}.
Random graphs have a short average path length. The graph is defined by the total number of nodes, and by the probability of any two nodes to be connected. They typically have a small clustering coefficient.

\subsubsection{Caveman Graph}

As defined by \citet*{W2003}.
The caveman structure is defined as having k isolated and fully connected `cliques' where one link is changed to connect one clique to another, rendering all cliques to be connected. Thus, relative to random graphs, they have a high clustering coefficient and a large average shortest path length.

\subsubsection{Small World Graph}

Defined by \citet*{WS1998}, and implemented by \citet*{BS2011}.
Small-world networks have characteristics that lie in between random graphs and highly clustered graphs (such as caveman graphs): they have a high clustering coefficient similar to the latter, but also have a small average shortest path similar to the former. Many real-world networks have been observed to have a small-world structure, and thus we included it in our simulations.

\subsubsection{Scale-Free Graph}

Defined by \citet*{BA1999}, and implemented by \citet*{BS2011}.
Scale-free network structures are often found where new nodes are constantly being added, and they are connected to already well-connected nodes. Such a structure displays a scale-free power-law distribution of the degree (connectivity) of nodes. Thus, there are few nodes that are highly connected, and more nodes that are moderately or mildly connected. Compared to random graphs, they have a smaller average shortest path.
In order to compare between different graph structures, the parameters were chosen such that the mean degree of the graphs were similar (approximately 30). For further details about parameter values, see Table ****

\subsection{Ideas}


After choosing a starting structure for our model, we then chose a distribution for the starting ideas (states) of nodes. The maximum number of different starting ideas is 100. Each node is randomly assigned one of these ideas. For the caveman structure, however, there were two other options: to either distribute the starting ideas `parallel' to the structure, i.e. such that all nodes in a cave shared the same idea, or `antiparallel' such that all nodes in a cave had a different idea. Why. As previously mentioned, each node had a small probability of adopting a novel idea from a virtually unlimited number of new ideas.