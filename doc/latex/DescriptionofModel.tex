%Description of the Model


The model used here is based on a study by \citet*{HN2006}.  Each simulation begins with a specified network structure as well as a distribution of the `idea' (or state) of the nodes. At each time step a node either changes its idea to that of one of its neighbours' ideas if its frequency surpasses a defined threshold, rewires to connect with a node that has the same idea, or generates a novel idea (this is the innovation parameter). 


Given the network structure and node states, three parameters are introduced: $\phi$ (probability of rewiring), $\alpha$ (probability of innovation), and $\delta$ (complex contagion threshold). As in \citet*{HN2006}, $\phi$ is a value from zero to one, and is the probability that one of the edges of a randomly chosen node \emph{i} will be changed to connect to another node \emph{j} that \emph{i} is unconnected with. We decided to add one more criterion to this definition: node \emph{j} is a node that has the same idea as node \emph{i}. This encourages the simulations to reflect a common tendency of individuals to seek out others who think like them. If there is no such node \emph{j}, then the chosen node will do nothing.


At each time step a node may `come up with a new idea' with a probability of $\alpha$. This value is small to reflect that novel ideas are not frequently observed.


We introduced a node threshold $\delta$ to the general model in order to investigate the behaviour of complex contagion as opposed to simple contagion. This was motivated by a study by \citet*{CM2007}. Simple contagion is well suited for modeling the spread of diseases since they may often be passed on by a single contact with an infected individual. However, as our intuition may suggest, other kinds of innovations raise questions about the legitimacy and credibility of the innovations themselves, and may thus require exposure to multiple sources of the innovation. This is called complex contagion. Models usually represent it in two ways regarding the number of connected sources that must have adopted an innovation in order to influence the agent (or individual) in question: it is either a fixed number (greater than one) or a fraction (between zero and one, inclusive). Our simulation used the second formulation since the degree of individual nodes varied across network structures.


\subsection{Networks}

For the purposes of our simulations, we used four network structures. \textbf{*****NETWORK STRUCTURE FIGURE****} illustrates them. These structures can be characterized by properties such as average shortest path lengths, clustering coefficients, and the degree of connectivity (see Table \ref{Tab2} for definitions). Below are short descriptions of each network structure. In order to compare between different structures, the network structure parame.ters were chosen such that the mean degree of the networks were similar (approximately 25). For further details about parameter values, see Table \ref{Tab1}.

\begin{table}[ht]
\caption{Table of parameters used in simulations.} % title of Table
\centering % used for centering table
\begin{tabular}{c c c c c} % centered columns (4 columns)
\hline\hline %inserts double horizontal lines
Network & Parameter & Value\\ [0.5ex] % inserts table
%heading
\hline % inserts single horizontal line
\,& $\phi$ & 0.1 & 0.3 & 0.5 \\
\,& $\alpha$ & 0.01 & 0.05 & 0.1\\
\,& $\delta$ & 0.001& 0.01 & 0.05\\
\,& $n$ & 1000\\
\,& $t_{end}$ & 1000\\
\hline
Caveman & $m$& 40 \\ % inserting body of the table
\,& $p$ & 40\\
\hline
Random (Erdos-Renyi) & $prob$ & 0.025 \\
\hline
Scale free & $m_0$ & 24\\
\,& $m_1$ & 12\\
\hline
Small world & $\kappa$& 24 \\
\,& $\beta$ & 0.1\\ [1ex] % [1ex] adds vertical space
\hline %inserts single line
\end{tabular}
\label{Tab1} % is used to refer this table in the text
\end{table}




\subsubsection{Random Graph}

Random graphs have a short average path length. The graph is defined by the total number of nodes, and by the probability of any two nodes to be connected. Thus, all connections are random. They typically have a small clustering coefficient. Here we used a variant of the Erd\H{o}s-R\'enyi random graph model \citep*{ER1960} as implemented by \citet*{BS2011}.

\subsubsection{Caveman Graph}

The caveman structure, as defined by \citet*{W2003}, has \emph{k} isolated and fully connected `cliques' from which one link is changed to connect one clique to another, rendering all cliques to be connected. Thus, relative to random graphs, they have a high clustering coefficient and a large average shortest path length.

\subsubsection{Small World Graph}

Small world graphs have characteristics that lie in between random graphs and highly clustered graphs (such as caveman graphs): they have a high clustering coefficient similar to the latter, but also have a small average shortest path similar to the former. Many real-world networks have been observed to have a small world structure, and thus we included it in our simulations. Here we used the graph as defined by \citet*{WS1998}, and implemented by \citet*{BS2011}.

\subsubsection{Scale-Free Graph}

Scale-free network structures are often found where new nodes are constantly being added, and they are connected to already well-connected nodes. Such a structure displays a scale-free power-law distribution of the degree (connectivity) of nodes. Thus, there are few nodes that are highly connected, and more nodes that are moderately or mildly connected. Compared to random graphs, they have a smaller average shortest path. This graph was implemented by \citet*{BS2011} as defined by \citet*{BA1999}.


\subsection{Ideas}

After choosing a starting structure for our model, we then chose a distribution for the starting ideas (states) of nodes. Each node was randomly assigned one of these ideas, thus allowing for multiple nodes to have the same idea. For the caveman structure, however, there were two other options: to either distribute the starting ideas `parallel' to the structure, i.e. such that all nodes in a cave shared the same idea, or `anti-parallel' such that all nodes in a cave had a different idea. This was used for the analysis of the effect of the idea distribution on the network structure. \textbf{****FIGURE OF IDEA DISTRIBUTION****} illustrates these idea distributions. Why was the caveman structure investigated? There were two reasons: firstly, it was straightforward how to define idea distributions that are in accord or disaccord with the caves in the network. Secondly, in the scientific community, research teams may often be made up of closely-connected members that are only weakly connected to other research teams, and within these teams, members may or may not be interested in the same ideas for research. 

As previously mentioned, each node had a small probability $\alpha$ of adopting a novel idea from a virtually unlimited number of new ideas.