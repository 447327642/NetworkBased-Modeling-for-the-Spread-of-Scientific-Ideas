%Description of the Model


The model used here is based on a study by \citet*{HN2006}.  Each simulation begins with a specified network structure as well as a distribution of the `idea' (or state) of the nodes. At each time step a node either changes its idea to that of one of its neighbours' ideas if its frequency surpasses a defined threshold, rewires to connect with a node that has the same idea, or generates a novel idea (this is the innovation parameter). 


Given the network structure and node states, three parameters are introduced: $\phi$ (probability of rewiring), $\alpha$ (probability of innovation), and $\delta$ (contagion threshold). As in \citet*{HN2006}, $\phi$ is a value from zero to one, and is the probability that one of the edges of a randomly chosen node \emph{i} will be changed to connect to another node \emph{j} that \emph{i} is unconnected with. We decided to add one more criterion to this definition: node \emph{j} is a node that has the same idea as node \emph{i}. This encourages the simulations to reflect a common tendency of individuals to seek out others who think like them.


At each time step a node may `come up with a new idea' with a probability of $\alpha$. This value is small to reflect that novel ideas are not frequently observed.


We introduced a node threshold $\delta$ to the general model in order to investigate the behaviour of complex contagion as opposed to simple contagion. This was motivated by a study by \citet*{CM2007}. Simple contagion is well suited for modeling the spread of diseases since they may often be passed on by a single contact with an infected individual. However, as our intuition may suggest, and as studies have shown, the spread of other kinds of innovations require several exposures before they are adopted by individuals; these situations refer to complex contagion.


\subsection{Networks}

<<<<<<< HEAD
For the purposes of our simulations, we used four network structures. \textbf{*****NETWORK STRUCTURE FIGURE****} illustrates them. These structures can be characterized by properties such as average shortest path lengths, clustering coefficients, and the degree of connectivity (see \textbf{******TABLE DEFINITIONS OF TERMS****} for definitions). Below are short descriptions of each network structure. In order to compare between different structures, the network structure parameters were chosen such that the mean degree of the networks were similar (approximately 30). For further details about parameter values, see \textbf{*******TABLE WITH PARAMETERS ****}.

\subsubsection{Random Graph}

A variant of the Erd\H{o}s-R\'enyi random graph model \citep*{ER1960}, and implemented by \citet*{BS2011}.
Random graphs have a short average path length. The graph is defined by the total number of nodes, and by the probability of any two nodes to be connected. Thus all connections are random. They typically have a small clustering coefficient.
=======
For the purposes of our simulations, we used four network structures. These structures can be characterized by properties such as average shortest path lengths, clustering coefficients, and the degree of connectivity. For a brief definition of these properties (which will be used in the results section), please see Table \ref{Tab2}. Below are short descriptions of each network structure.

\begin{table}[ht]
\caption{List of terminology} % title of Table
\centering % used for centering table
\begin{tabular}{c c c c } % centered columns (4 columns)
\hline\hline %inserts double horizontal lines
 & Term\\ [0.5ex] % inserts table
%heading
\hline % inserts single horizontal line
Neighborhood index & The fraction of the holders of the same idea\\
\,&  who are neighbor as well averaged over all ideas.\\
\hline
Intra-idea distance& The average distance between \\
&holders of the same idea in the network.\\
\hline
Dominant frequency & The frequency of the dominant idea\\
& in the network at each time steps of the
simulation.\\
\hline
Average dominance time& The average number of time steps \\
&in which the dominant idea keeps its
dominance.\\
\hline
Novelty index& The fraction of newly generated ideas.\\
\hline
Average shortest path & The average number of steps\\
& along the shortest paths for all possible pairs of network nodes.\\
\hline
Clustering coefficient &  A measure of degree to which\\
& nodes in a graph tend to cluster together.\\
\hline
Degree of connectivity &  The number of edges incident to the vertex.\\
\hline
Connected component & A subgraph in which any two vertices are\\
 &connected to each other by paths, and which is connected to no\\
  &additional vertices in the supergraph.\\
\hline
Diameter of network& The longest of all the calculated shortest paths\\
& in a network.\\
\\ [1ex] % [1ex] adds vertical space
\hline %inserts single line
\end{tabular}
\label{Tab2} % is used to refer this table in the text
\end{table}



\subsubsection{Random Graph}

The second variant of the Erd\H{o}s-R\'enyi random graph model \citep{ER1960} as mentioned in \citet{Wiki}, and implemented by \citet*{BS2011}.
Random graphs have a short average path length. The graph is defined by the total number of nodes, and by the probability of any two nodes to be connected. They typically have a small clustering coefficient.
>>>>>>> I put here the matlab codes in this folder to be able to put them in the report, later I'll figure it out how remove them

\subsubsection{Caveman Graph}

As defined by \citet*{W2003}.
<<<<<<< HEAD
The caveman structure is defined as having \emph{k} isolated and fully connected `cliques' from which one link is changed to connect one clique to another, rendering all cliques to be connected. Thus, relative to random graphs, they have a high clustering coefficient and a large average shortest path length.
=======
The caveman structure is defined as having $k$ isolated and fully connected `cliques' where one link is changed to connect one clique to another, rendering all cliques to be connected. Thus, relative to random graphs, they have a high clustering coefficient and a large average shortest path length.
>>>>>>> I put here the matlab codes in this folder to be able to put them in the report, later I'll figure it out how remove them

\subsubsection{Small World Graph}

Defined by \citet*{WS1998}, and implemented by \citet*{BS2011}.
Small world graphs have characteristics that lie in between random graphs and highly clustered graphs (such as caveman graphs): they have a high clustering coefficient similar to the latter, but also have a small average shortest path similar to the former. Many real-world networks have been observed to have a small world structure, and thus we included it in our simulations.

\subsubsection{Scale-Free Graph}

Defined by \citet*{BA1999}, and implemented by \citet*{BS2011}.
Scale-free network structures are often found where new nodes are constantly being added, and they are connected to already well-connected nodes. Such a structure displays a scale-free power-law distribution of the degree (connectivity) of nodes. Thus, there are few nodes that are highly connected, and more nodes that are moderately or mildly connected. Compared to random graphs, they have a smaller average shortest path.
<<<<<<< HEAD
=======
In order to compare between different graph structures, the parameters were chosen such that the mean degree of the graphs were similar (approximately 30). For further details about parameter values, see Table \ref{Tab1}.

\begin{table}[ht]
\caption{Table of parameters used in simulations.} % title of Table
\centering % used for centering table
\begin{tabular}{c c c c c} % centered columns (4 columns)
\hline\hline %inserts double horizontal lines
Network & Parameter & Value\\ [0.5ex] % inserts table
%heading
\hline % inserts single horizontal line
\,& $\phi$ & 0.1 & 0.3 & 0.5 \\
\,& $\alpha$ & 0.01 & 0.05 & 0.1\\
\,& $\delta$ & 0.001& 0.01 & 0.05\\
\,& $n$ & 1000\\
\,& $t_{end}$ & 1000\\
\hline
Caveman & $m$& 40 \\ % inserting body of the table
\,& $p$ & 40\\
\hline
Random (Erdos-Renyi) & $prob$ & 0.025 \\
\hline
Scale free & $m_0$ & 24\\
\,& $m_1$ & 12\\
\hline
Small world & $\kappa$& 24 \\
\,& $\beta$ & 0.1\\ [1ex] % [1ex] adds vertical space
\hline %inserts single line
\end{tabular}
\label{Tab1} % is used to refer this table in the text
\end{table}



>>>>>>> I put here the matlab codes in this folder to be able to put them in the report, later I'll figure it out how remove them

\subsection{Ideas}

After choosing a starting structure for our model, we then chose a distribution for the starting ideas (states) of nodes. Each node was randomly assigned one of these ideas, thus allowing for multiple nodes to have the same idea. For the caveman structure, however, there were two other options: to either distribute the starting ideas `parallel' to the structure, i.e. such that all nodes in a cave shared the same idea, or `antiparallel' such that all nodes in a cave had a different idea. This was used for the analysis of the effect of the idea distribution on the network structure. \textbf{****FIGURE OF IDEA DISTRIBUTION****} illustrates these idea distributions. Why was the caveman structure investigated? There were two reasons: firstly, it was straightforward how to define idea distributions that are in accord or disaccord with the connections of nodes in the network. Secondly, in reality research teams are often made up of closely-connected members that are only weakly connected to other research teams, and within these teams, members may or may not be interested in the same ideas for research. 

As previously mentioned, each node had a small probability $\alpha$ of adopting a novel idea from a virtually unlimited number of new ideas.
