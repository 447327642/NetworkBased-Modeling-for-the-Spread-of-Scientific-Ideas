%Introduction and Motivation

We live in a time in which aspects of our lives and of our world become more and more connected with each other. To understand one aspect, we must understand many other aspects, which all together form a large, complex system, or network. Globalization has changed the meaning of `distance' and communication, allowing seemingly unrelated and unconnected individuals to share more than they ever could before. Fields of studies are overlapping with each other, creating new interdisciplinary domains and building a diverse playground for the sharing of ideas. But how do ideas spread? This question is especially interesting with the increase of technology that allows us to record and visualize the networks that connect individuals and ideas, and in particular the ability to `see' how they change. This can lead to insight about why and when ideas spread and into the complexity of the matter. Research has shown that not only does the nature of information or innovation influence the diffusion of it, but also that the structure of a network influences the diffusion dynamics. Here, we try to simulate the spread of scientific ideas in different networks. The model presented is based on two studies: one that investigated critical parameter values for complex contagion \citep{CM2007} and another that investigated critical values of a rewiring parameter \citep{HN2006}.

\subsection{Fundamental Questions}

The main goals of this simulation study are to investigate how network structure influences the distribution of ideas, and how the distribution of ideas influences network structure. For a list of the terminology that will be used throughout the paper, please refer to ****TABLE DEFINITION OF TERMS*****. By varying three parameters - probability of rewiring, rate of innovation, and complex contagion threshold ($\phi$, $\alpha$, and $\delta$ respectively) - and using different network structures and idea distributions, we observed how network structure characteristics changed. Similarly, we observed how the distribution of ideas and the connections between them changed. Below we describe our questions more specifically.


\subsubsection{Effects of Network Structure on Idea Distribution}

Given a starting network and a random idea distribution, how do different network structures affect the distance between nodes that have the same idea (intra-idea distance)? How do they affect the neighbourhood index? How do they change the emergence of dominant ideas and their time of dominance? How do their effects depend on the values of $\phi$, $\alpha$ and $\delta$?


More rigid network structures (those with less `randomness', such as the caveman and the small world networks) may make it more difficult for `like-minded' nodes (that is, nodes with the same idea) to connect and may thus have smaller neighbourhood indexes and larger intra-idea distances than the more random network structures (such as the random and scale-free networks). Their effects may be more sensitive to the values of $\phi$ (because this affects how likely it is for their structure to change) and to values of $\delta$ because being restricted to a more closed group of nodes makes it difficult to reach a threshold necessary to become similar to surrounding nodes. Values of $\alpha$ may decrease the neighbourhood indexes by creating larger diversity among neighbouring nodes.

If more rigid network structures do make it more difficult for like-minded nodes to connect, then it would be more difficult for a dominant idea to emerge in these networks. These effects may be smaller for larger values of $\phi$ since these values would allow for the structure to change more. For larger values of $\phi$ therefore one could expect that the effects of the network structures on the characteristics of the idea distribution are more similar since allowing to change the structure removes their initial influence.


\subsubsection{Effects of Idea Distribution on Network Structure}

Given a starting idea distribution and a caveman network structure, how do different idea distributions affect the average path length and diameter of the network? Do they change the number of connected components in the network? Do clusters form differently, and how does the clustering coefficient change? What does the distribution of node degree look like? How do these effects depend on the values of $\phi$, $\alpha$, and $\delta$?


If the starting idea distribution is parallel to the caveman network structure (see ******TABLE DEFINITIONS OF TERMS*** for definitions), then like-minded nodes will already be connected and thus rewiring will probably not change much of the average path length, nor will it change the diameter. Similarly, the clustering coefficient will remain high just like the starting value. The distribution of the node degree will also not change (nodes will have one of two values for their degree). In other words, if the idea distribution is parallel to the network structure, the structure will not change much. Changing $\phi$ and $\delta$ will not change these effects, and perhaps increasing $\alpha$ will decrease the clustering coefficient and will increase the number of connected components because nodes will disconnect from nodes with novel ideas and will rewire to nodes with the same idea.

If the starting idea distribution is random, then the network's caves will disintegrate as nodes will rewire with other nodes outside of their caves. This will change the degree distribution by increasing its variance (nodes will have a variety of different degree values). Depending on the value of $\delta$ this disintegration may be reduced because nodes have a higher chance of forming dominant ideas within caves. Similarly, increasing $\phi$ will increase the disintegration of caves. Thus for this idea distribution the parameter values may play a larger role.

If the starting idea distribution is anti-parallel, nodes within each cave will initially be connected with nodes that do not hold the same idea as them. Therefore the threshold $\delta$ will not be met in order for nodes to change their ideas, and they will rewire with other nodes outside of their cave. The clustering coefficient and the number of connected components will likely decrease, and the diameter and the average path length will decrease as well since the structure will change significantly. The degree distribution will increase in variance. Increasing $\phi$ and $\alpha$ and decreasing $\delta$ will probably increase the magnitude of these effects. Thus, having an anti-parallel idea distribution will probably display the most changes in the characterstics of the network structure that are in question.



