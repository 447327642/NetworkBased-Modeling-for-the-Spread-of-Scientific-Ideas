%Introduction and Motivation

We live in a time in which aspects of our lives and of our world become more and more connected with each other, and therefore they tend to become more complex. Globalization has changed the meaning of `distance' and communication, allowing seemingly unrelated and unconnected individuals to share more than they ever could before. Fields of studies are overlapping with each other, creating new interdisciplinary domains and building a diverse playground for the sharing of ideas. But how do ideas spread? This question is especially interesting with the increase of technology that allows us to record and visualize the networks that connect individuals and ideas, and in particular the ability to `see' how they change. This can lead to insight about why and when ideas spread and into the complexity of the matter. Research has shown that not only does the nature of information or innovation influence the diffusion of it, but also that the structure of a network influences the diffusion dynamics. The model presented in this simulation study is based on two studies: one that investigated critical parameter values for complex contagion \citep{CM2007} and another that investigated critical values of a rewiring parameter \citep{HN2006}.

\subsection{Fundamental Questions}

The main goals of this simulation study are divided into two is to investigate how network structure influences the distribution of ideas, and how the distribution of ideas influences network structure. By varying three parameters - probability of rewiring, rate of innovation, and complex contagion threshold ($\phi$, $\alpha$, and $\delta$ respectively) - and using different network structures and idea distributions, we observed how network structure characteristics changed. Similarly, we observed how the distribution of ideas and the connections between them changed. Below we described our questions more specifically.


\subsubsection{Effects of Network Structure on Idea Distribution}

Given a starting network and a random idea distribution, how do different values of $\alpha$ and $\delta$ change the neighbourhood index of nodes with the same idea? For which combinations of parameters does the distance between nodes holding the same idea (intra-idea distance) decrease, if at all? Do these parameters change the emergence of dominant ideas (ideas held by most of the nodes) and their duration of dominance? Do all of these characteristics behave the same way for different starting network structures? 

Large values of $\delta$ may prevent nodes from accepting neighbouring ideas and may result in a decrease in the neighbourhood index and in the intra-idea distance because it could prevent the formation of homogeneous groups. Additionally, the inclusion of large enough values of $\alpha$ may destabilize the formation of groups because already-established groups may lose nodes that share the same idea, and may thus also decrease the neighbourhood index. The effects may be more visible for the caveman structure because this structure already restricts most nodes to a certain set of neighbours and with a too large $\delta$ may never change their opinions.

Previous studies using complex contagion thresholds $\delta$ and two idea states for nodes showed that for large enough $\delta$ the randomization of the network ties did not help spread ideas more quickly throughout the network \citep*{CM2007}. However, in our study the rewiring is not completely random: nodes are connected to other nodes sharing the same idea. Thus, the more random network structures (random graph, small world) may benefit more from their random ties and  have a larger frequency of dominance for the dominant idea as well as a longer duration of dominance. Also, the more dense networks (e.g. those with shorter average path lengths) might reach the states of dominant ideas more frequently than the less dense networks because the $\delta$ threshold is more likely to be satisfied. *******


\subsubsection{Effects of Idea Distribution on Network Structure}

Given a starting idea distribution and a caveman network structure, how do different values of $\phi$, $\alpha$, and $\delta$ change the average path length, diameter, and number of connected components in the network? Do clusters form, and how does the clustering coefficient change? What does the distribution of node degree look like? Do these changes vary with the starting idea distribution?

For larger values of $\phi$ there may be larger average path lengths and more connected components because nodes will connect to similar nodes more often and will cluster with each other (also increasing the clustering coefficient). This would also increase the diameter of the network. If the starting idea distribution is parallel to the caveman structure, the caves will remain clustered and may completely separate from other caves. Increasing $\alpha$ however may reduce both the clustering and the complete separation of the caves if it is high enough. If the starting idea distribution is antiparallel, this will encourage new clusters to form and they will also be completely separated, but they will not form as quickly. For a starting idea distribution that is random, it might depend more on the starting circumstances of a neighbourhood: if there is a neighbourhood of $\delta$*connectivity nodes that share the same idea, this may be enough to form a cluster than will become the connected component. However, it is more likely that many connected components will arise. 

Degree distribution...


