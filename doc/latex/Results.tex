%Results and Discussion


\subsection{Effects of Network Structure on Idea Distribution}

For this analysis, four network structures were implemented (caveman, small world, random, and scale-free) with a random idea distribution. Simulations were run for each network structure for 27 different parameter combinations (three values for each of $\phi$, $\delta$ and $\alpha$). The effects of each network structure on the idea distribution was evaluated on five features: the neighbourhood index, the intra-idea distance, the frequency of dominance of an idea, the average dominance time for an idea, and the novelty index. We also investigated how these effects vary with the three parameters ($\phi$, $\delta$, $\alpha$).

\subsubsection{Neighbourhood Index and Intra-Idea Distance}

For any parameter combination, the caveman and small world structures resulted in larger intra-idea distances (respectively) than those of the random and scale-free structures, which were very similar to each other (***FIGURE 1.1****). Additionally, a similar influence was found on the neighbourhood index: the caveman structure held the largest index regardless of parameter combination, followed by the small world, random, and scale-free structures (****FIGURE 1.2****). It seems that the caveman structure encourages nodes to be within the direct neighbourhood of like-minded nodes (nodes with the same idea) and at a farther distance from like-minded nodes that are not in their direct neighbourhood, whereas the random and scale-free structures have a tendency to keep like-minded nodes in each other's direct neighbourhood but to also keep those like-minded nodes not in their direct neighbourhood at a shorter distance.

\subsubsection{Frequency of Dominance}

\subsubsection{Average Dominance Time and Novelty Index}


Because the results of the average dominance time and the novelty index were too dependent on the parameter values, we did not include their results. It is not surprising th

In general,   
four structures, random idea distribution, three $\phi$ three $\delta$ and three $\alpha$ --> 27 combinations for each structure. five features

neighbourhood index
	larger for caveman and small world
	smaller for random and scale-free
intra-idea distance
	larger for caveman and small world
	random and scale-free are the same and lower
dominant frequency
	for all structures and for all parameter combinations the frequency increased --> adoption of dominant idea.
	more rapid increase for caveman. but could be stochastic and dependent on parameter combination
average dominance time
	too dependent on parameter values
	not dependent on alpha.
novelty index
	too dependent on parameter values
	dependent on alpha obviously.

alpha
	for all structures: increasing it will decrease the intra-idea distance. !!! Could be because of more single unique ideas that have no 'distance'. more visible effect for random network and scale-free
	for all structures: increasing it decreased neighbourhood index. more visible effect for random and scale-free.
phi
	for all structures: increasing it will decrease the intra-idea distance. more visible effect for caveman and small world.
	for neighbourhood index: in random and scale-free increasing phi increased it, and in caveman increasing phi decreased it, and for small world no effect on index.
delta
	for all structures: increasing it decreased intra-idea distance. look at plot: certain spikes - combinations
	for neigbourhood index: in caveman increasing delta increases index. for other structures no connection between delta and index. obvious why it would, but not clear why only for caveman

caveman and small world keep nodes with same idea close, but those wiht same idea that are not in direct neighbourhood are far (as compared with random and scale-free).





\subsection{Effects of Idea Distribution on Network Structure}

three idea distributions on caveman network, three $\phi$ three $\delta$ and three $\alpha$ --> 27 combinations. examining five features of structure

clustering coefficient
	higher for parallel
degree distribution
	for parallel ended up similar to scale-free graph distribution
	for random and antiparallel the distribution was similar to that of a random graph
connected components
	all the same: 1. always all reachable
average path length
	larger for parallel
diameter
	larger for parallel


alpha
	no effect
phi
	same effect for all distributions: increasing it will decrease path length, diameter, clustering coefficient. less effect for parallel
delta
	for parallel only: increasing it increased the clustering coefficient.

Parallel is different to random and antiparallel: 
- diameter, average shortest path and clustering coefficient are HIGHER than other two distributions'.
