<<<<<<< HEAD
%Results and Discussion


\subsection{Effects of Network Structure on Idea Distribution}

For this analysis, ideas were assigned randomly onto the nodes of four different network structures (caveman, small world, random, and scale-free). Simulations were run for each network structure for 27 different parameter combinations (three values for each of $\phi$, $\delta$ and $\alpha$). The effects of each network structure on the idea distribution was evaluated on five features: the intra-idea distance, the neighbourhood index, the frequency of dominance of an idea, the average dominance time for an idea, and the novelty index. We also investigated how these effects vary with the three parameters ($\phi$, $\delta$, $\alpha$).

We observed that the caveman and small world network structures had a similar influence on the intra-idea distance and neighbourhood index, and the random and scale-free network structures had a different - yet similar to each other - influence on these features.

\subsubsection{Intra-Idea Distance and Neighbourhood Index}

For any parameter combination, the caveman and small world structures resulted in larger intra-idea distances (respectively) than those of the random and scale-free structures, which were very similar to each other (***FIGURE 1.1****). Additionally, a similar influence was found on the neighbourhood index: the caveman structure held the largest index regardless of parameter combination, followed by the small world, random, and scale-free structures (****FIGURE 1.2****). It seems that the caveman structure encourages nodes to be within the direct neighbourhood of like-minded nodes (nodes with the same idea) and at a farther distance from like-minded nodes that are not in their direct neighbourhood, whereas the random and scale-free structures have a tendency to keep like-minded nodes in each other's direct neighbourhood but to also keep those like-minded nodes not in their direct neighbourhood at a shorter distance. This could be a result of the general larger average path distance that caveman networks have when compared to random and scale-free networks.

\paragraph{Dependence on $\alpha$}
For all network structures, increasing the level of innovation $\alpha$ decreased the intra-idea distance. This effect was more pronounced in the scale-free and random networks (***FIGURE 1.5-1.8****). Does innovation bring people together in scienitific communities? Perhaps it does not: under complex contagion, nodes that adopted a novel idea could not influence other nodes since they were the sole holders of these ideas. Therefore, their intra-idea distance for these `novel nodes' would have been zero. 

Higher values of $\alpha$ also increased the neighbourhood index for all network structures. This correlation was again most visible for the scale-free network, followed by the random network (***FIGURE 1.9-1.12***). One possible explanation for this correlation is that the more novel ideas nodes create, the less likely it is that the contagion threshold is met for other ideas, and thus nodes will only be rewiring instead of also changing their ideas. Thus more like-minded nodes will be connected, and the index increases. 


\paragraph{Dependence on $\phi$}
By increasing $\phi$, the intra-idea distance decreased for all network structures. This correlation is quite an intuitive result since $\phi$ is the probability of deleting a connection between two nodes with different ideas and the formation of a new connection between two nodes with the same idea, and thus, by increasing $\phi$ the distance between like-minded nodes decreases. Unlike the $\alpha$ parameter, the effects of $\phi$ are more pronounced for the caveman and small world structures rather than for the random and scale-free networks (*****FIGURES 1.13-1.16****).

On the other hand, th effects of $\phi$ on the neighborhood index varied between networks. Increasing $\phi$ increased the neighbourhood indexes of the random and scale-free networks, while it decreased the neighbourhood index of the caveman network and had no correlation with changes in the small world network (***FIGURES 1.17-1.20***).

\paragraph{Dependence on $\delta$}
Increasing values of the complex contagion threshold $\delta$ decreased the intra-idea distance for all network structures (****FIGURE 1.21-1.24***). Using the $\delta$ value of 0.05 resulted in slightly lower intra-idea distances than the two smaller values (which behaved very similarly) for the caveman and small world network structures. WHY?

The neighbourhood index for the caveman network increased as $\delta$ increased (****FIGURE 1.25***). This is intuitive since requiring more like-minded nodes to be in the direct neighbourhood for a node to adopt their idea automatically increases the number of like-minded nodes in the neighbourhood (neighbourhood index). However, increasing $\delta$ could have had the opposite effect: if the threshold was too high, nodes would not have adopted their neighbhours' ideas and thus the neighbourhood index would have remained small. No correlation was found between the neighbourhood index of the small world, scale-free, and random network structures and values of $\delta$. WHY only for caveman?

\subsubsection{Frequency of Dominance}

This feature is interesting if one considers scientific society. How dominant are dominant ideas in the scientific community? Does the structure of this community influence this dominance? For all parameter combinations and for all network structures in our simulations, the frequency of dominance of ideas increased with time. This may suggest that none of these structures impede the adoption of new ideas, and that, not surprisingly, society tends to adopt dominant ideas. More surprisingly, however, the increase in dominance frequency progressed more quickly for the caveman network structure (***FIGURE 1.3***). Could it be that this structure encourages nodes to adopt dominant ideas more easily? This may not be generalizable because the results are quite sensitive to parameter values due to the stochasticity of the simulations. For example, ****FIGURE 1.4*** shows a different combination of parameters, and here the faster increase in the dominance is not observed for the caveman structure.


\subsubsection{Average Dominance Time and Novelty Index}


Because the results of the average dominance time and the novelty index were too dependent on the parameter values, we did not include their results. The network structure does not seem to play a strong enough role over all parameter combinations in influencing these features. It is not surprising, however, that the novelty index was highly dependent on the value of $\alpha$, but it is not so intuitive why the average dominance time was not correlated with $\alpha$ at all: increasing the number of novel ideas decreases the possible number of 'followers' for already-established ideas. Perhaps the value of $\alpha$ would need to be increased to observe this behaviour.



\subsection{Effects of Idea Distribution on Network Structure}


Here we investigated the opposite direction of effects: how changing the idea distribution affects the resulting network structure. To do this we applied three different idea distributions (random, parallel, and antiparallel to the structure) to a caveman network. There were again 27 different parameter combinations for each idea distribution, as in the previous analysis. The five features of the network structure that we evaluated were: the clustering coefficient, the degree distribution of the nodes, the number of connected components, the average path length, and the network diameter. 

We observed that the influence of applying a parallel idea distribution on the features of the network structure was quite different to the influence of the random and antiparallel idea distributions. While the networks always remained fully connected (**FIGURE 2.4****), the remaining features changed. The parameter $\alpha$ did not change these effects for any of the idea distributions (see ***FIGURES 2.6-2.14***). Given these simulation results, perhaps a caveman-structured scientific community would also manage certain levels of innovativity without changing its fundamental structure.


\subsubsection{Clustering Coefficient}

The clustering coefficient of resulting networks varied with the type of idea distribution. When nodes in the same cave shared the same idea (parallel distribution), the clustering coefficient was larger (***FIGURE 2.3***). Additionally, the clustering coefficients for both the random and the anti-parallel idea distributions decreased in a similar manner regardless of the parameter combinations. Both of these results are intuitive since less rewiring would have taken place for the parallel distribution case, and the high clustering coefficient of the caveman structure would have been conserved, whereas more rewiring would have ocurred for the two other distributions, thus decreasing the clustering coefficients.

\paragraph{Dependence on Parameters}
Increasing values of the rewiring parameter $\phi$ decreased the clustering coefficient for all starting distributions, especially for the random and antiparallel distributions (***FIGURES 2.21-2.23***). This again is intuitive since $\phi$ increases the chances of changing connections in a highly clustered network.
Increasing values of $\delta$, however, only increased the clustering coefficient when using a parallel idea distribution, and only slightly (***FIGURES 2.30-2.32***). WHY?

\subsubsection{Average Path Length}
The average path length of resulting networks was larger when a parallel idea distribution was used (***FIGURE 2.1****). This is not surprising, since the structure of the caveman network was more conserved (because most nodes were already connected to like-minded nodes and did not need to rewire), and its average path length is larger than that of more randomized networks, such as the ones resulting from a larger amount of rewiring. 
\paragraph{Dependence on Parameters}
Increasing $\phi$, regardless of the idea distribution, decreased the average path length. This effect was more prononced for the random and anti-parallel idea distribution cases (**FIGURE 2.15-2.17***). This is another intuitive result since more rewiring naturally disturbs the rigid structure of a caveman network, thereby decreasing its large average path length; rewiring also occurs less frequently in the parallel idea distribution case because most nodes are already connected to like-minded nodes.
Changing values of $\delta$ (which requires more than one neighbouring node to have the same idea in order to influence the chosen node) did not change the effects of idea distributions on the average path length (***FIGURES 2.24-2.26***). This is natural, since nodes were either already connected to a significant number of nodes with the same idea (in the case of the parallel distribution), not connected to any other node with the same idea (the antiparallel case), or connected to nodes with random assignment of ideas. Thus the threshold $\delta$ would either already be fulfilled from the start, would definitely not be fulfilled, or would have a very small chance of being fulfilled, respectively.


\subsubsection{Network Diameter}
Similar to the clustering coefficient and the average path length, the network diameter was larger when the parallel idea distribution was applied (***FIGURE 2.2*****). This distribution encouraged the structure of the caveman network to remain mostly unchanged, and thus the farthest distance between two nodes was larger than for the case of random or anti-parallel distributions, where more rewiring occured. 
\paragraph{Dependence on Parameters}
As $\phi$ increased, the network diameter decreased for all idea distributions, and slightly less for the parallel distribution (**FIGURES 2.18-2.20*****). Rewiring a caveman network intuitively may decrease the network diameter by connecting more of its caves.
Similar to the average path length, values of $\delta$ did not change the behaviour of the network diameter given the idea distributions (***FIGURES 2.27-2.29***).


\subsubsection{Degree Distribution}
Random graphs have a somewhat normally distributed degree distribution. Scale-free graphs, on the other hand, have a degree distribution that follows a scale-free power-law. We observed that the degree distribution of the networks depended on the idea distribution (***FIGURE 2.5****). A parallel idea distribution resulted in a degree distribution similar to that of a scale-free graph, which is not surprising. Caveman graphs have two or three different degrees for their nodes, and allowing for some rewiring would 'smooth' out this discrete distribution. Similar to previous results, the random and antiparallel idea distributions behaved similarly: their degree distribution was similar to that of random graphs. It is interesting to see such a visible difference in these distributions over relatively few time steps (1000 steps), regardless of the parameter combinations. 


\subsection{Discussion}


Expected results vs actual results.
network structure affects the ditirbution of ideas --> distribution of scientific idea depends on structure of scientific community
delta effect with caveman on index
random and antiparallel more dependent on delta
distribution changed a lot, but the others didn'ttoo much.



Future Study
intra-idea distance = dont' include ideas that are held by only one node.
=======
\subsection{Phase 1:}
In this stage of our study, the influence of network structure on idea distribution is assessed. To do so, five different features of the idea distribution will be measured for four different network structures at 27 different parameter sets.\\
Four different network structures are:
\begin{itemize}
	\item	Caveman Network
	\item Random Network
	\item Scale free Network
	\item Small world Network
\end{itemize}

And five different features of the idea distributions are:
\begin{itemize}
	\item Neighborhood index: The fraction of the holders of the same idea who are neighbor as well averaged over all ideas.
	\item Intra idea distance: The average distance between holders of the same idea in the network.
	\item Dominant frequency: The frequency of the dominant idea in the network at each time steps of the simulation.
	\item Average dominance time: The average number of time steps in which the dominant idea keeps its dominance.
	\item Novelty index: The fraction of newly generated ideas.
\end{itemize}

For 27 combinations of the following three parameters:
\begin{itemize}
	\item Alpha: The probability that a random person in the network comes up with a novel idea at each time step of the simulation.
	\item Phi: The probability that a randomly chosen person deletes its connection with a holder of an alternative idea and makes a new connection with a holder of the same idea. The complement of phi (1-phi) is the probability that a randomly chosen person changes its idea to that of one of its neighbors.
	\item Complex contagion threshold: In order for a randomly chosen person to change its idea to that of one of its neighbors, the fraction of the neighbors holding that idea should exceed a threshold called complex contagion threshold.  
\end{itemize}

In this stage, initially the same random idea distribution is used for each simulation. Moreover the average degree of each network is the same (around 25).

\bigskip

\textbf{Results:}\\

Regardless of parameter choice, we observed that intra idea distance for Caveman network is higher than all and second highest is Small world network. However, intra idea distance for Random network and Scale free network are very similar and significantly lower than that of Caveman and Small world network (figure 1-left) which indicates that in Small world network and Caveman network, persons with the same idea are dispersed in the network while in Random network and scale free networks they are closer to each other. Additionally, the same order is roughly valid for neighborhood index (Caveman>Small world>Random>Scale free) which implies that the holders of the same idea have more tendency to be the neighbor of each other in the Caveman network than other networks (figure 1-right). Therefore, Caveman network has more tendency to keep the holders of the same idea in direct neighborhood, on the other hand it allows the holders of the same idea which are not in direct neighborhood to be very far from each other, whereas Scale free or Random networks have less tendency to keep the agents with the same idea in direct neighborhood, but they restrict the holders of the same idea who are not in direct neighborhood to be in a shorter distance from each other.
The influence of network structure on average dominance time and novelty index was strongly parameter dependent and noisy so it is not interpretable (data not shown). Nevertheless, as was mentioned above, two features of the idea distribution are clearly influenced by network structure which support the idea that the distribution scientific idea depends on the structure of the scientific society.   


  Figure 1: The influence of network structure on intra idea distance (left) and neighborhood index (right) for 27 distinct sets of parameters (triplets of alpha-phi-threshold).                                           

Another feature of idea distribution whose investigation is of importance especially in the scientific society is the frequency of dominant idea. Thus, we studied also the influence of network structure on the frequency of the dominant idea in different time steps of the simulation and for different sets of alpha-phi-threshold parameters. We observed that for each parameter sets and for every network structures gradually the frequency of the dominant idea increases which reflects the tendency of society to adopt the dominant idea.
More importantly and less intuitively, we observed that for some network structures the increase in the frequency of the dominant idea is more rapid than others. For example the increase in the frequency of the dominant idea for Caveman network structure is significantly faster than other network structures, which implies that in societies with Caveman network structure people have more tendency to adopt the major idea while dominant idea will not be as strongly supported in other network structures (figure 2-top). However, this is not an absolutely general result since due to the inherent stochasticity of the simulation, the results are quite parameter dependent and in different parameter sets we observe some deviation from the general superiority of Caveman network. For example, in figure2-bottom which corresponds to a different parameter set, at the later stages of the simulation the frequency of dominant idea for Scale free network surpassed that of Caveman network structure.



Figure 2: Frequency of the dominant idea at each time for four distinct network structures and two different sets of parameters: phi=0.5, alpha=0.01, threshold=0.05 (top) and phi=0.3, alpha=0.1, threshold=0.05 (bottom).
Next, we studied the dependence of the features of idea distribution on parameters. In each study we keep two parameters constant and one is variable. For instance when we want to study the influence of alpha on a feature, for each of the three values of alpha, we�ll measure the feature for 3?3=9 distinct pairs of the other two parameters, so we�ll obtain 3 vectors of length 9 for each study.

Dependence on alpha:

By increasing alpha, the intra idea distance decreases for every network structure which implies that increasing creativity of a scientific society correlates with the decrease in the distance between agents with the same idea. Thus, in creative scientific societies holders of the same idea will tend to be closer to each other. Moreover, the dependence on alpha is more significant for Random network and Scale free than Caveman and Small world networks, therefore Random network and Scale free network are more influenced by the creativity of the scientific society (Figure 3).
On the other hand, by increasing alpha the neighborhood index increases. This positive correlation implies that in creative societies holders of the same idea are rarely placed in direct neighborhood. Again this dependence is more profound for Random network and scale free network (Figure 4).
Furthermore, obviously the novelty index depends on the alpha for every network structures. However, the there is no correlation between average dominance time and alpha.   

 
 
Figure 3: Dependence of intra idea distance on alpha for each network structure. Horizontal axis corresponds to distinct pairs of phi-threshold and   vertical axis is the intra idea distance. Each curve corresponds to a different value of alpha. Black curve corresponds to the smallest value of alpha and it always has the highest intra idea distance while the red curve corresponds to the largest value of alpha which always has the lowest value of intra idea distance. 
 
 
Figure 4: Dependence of neighborhood index on alpha for each network structure. Horizontal axis corresponds to distinct pairs of phi-threshold and   vertical axis is the neighborhood index. Each curve corresponds to a different value of alpha. Red curve corresponds to the largest value of alpha and it has the highest neighborhood index while the black curve corresponds to the smallest value of alpha which has the lowest value of intra idea distance. 
Dependence on Phi:
By increasing phi, the intra idea distance decreases for every network structure. This correlation is quite an intuitive result since phi is the probability of deleting a connection between two persons with different idea and formation of a new connection between two persons with the same idea, so by increasing phi the distance between agents with the same idea decreases. Unlike alpha parameter, dependence on phi is more significant for Caveman and Small world networks than Random and Scale free networks (Figure 5).
On the other hand, dependence of neighborhood index on phi is very network dependent. For Random and Scale free networks, by increasing phi, the neighborhood index increases while for Caveman network, increasing phi correlates with decreasing neighborhood index and for Small world network there is no significant dependence of neighborhood index on phi (Figure 6).








Figure 5: Dependence of intra idea distance on phi for each network structure. Horizontal axis corresponds to distinct pairs of alpha-threshold and   vertical axis is the intra idea distance. Each curve corresponds to a different value of phi. Black curve corresponds to the smallest value of phi and it always has the highest intra idea distance while the red curve corresponds to the largest value of phi which always has the lowest value of intra idea distance.



 
 
Figure 6: Dependence of neighborhood index on phi for each network structure. Horizontal axis corresponds to distinct pairs of alpha-threshold and   vertical axis is the neighborhood index. Each curve corresponds to a different value of phi. Increasing phi leads to increase neighborhood distance for Random and Scale free networks while increasing phi causes the neighborhood index to be decreased for Cavemen Network. For Small world network there is no significant dependence of neighborhood index on phi. 
Dependence on complex contagion threshold:
By increasing complex contagion threshold, the intra idea distance for all the four networks reduced. The intra idea distance for 0.001 and 0.01 thresholds are pretty the same for Caveman network and Small world network while for random network and scale free network the intra idea distance for these two thresholds are significantly different (Figure 7). 
For Caveman network, increasing complex contagion correlates with increasing neighborhood index while for other network structures no significant dependence of neighborhood index on complex contagion threshold was observed (figure 8).

 
Figure 7: Dependence of intra idea distance on complex contagion threshold for each network structure. Horizontal axis corresponds to distinct pairs of alpha-phi and   vertical axis is the intra idea distance. Each curve corresponds to a different value of complex contagion threshold. Black curve corresponds to the smallest value of complex contagion threshold and it has the highest intra idea distance while the red curve corresponds to the largest value of complex contagion threshold which has the lowest value of intra idea distance.

Figure 8: Dependence of neighborhood index on complex contagion threshold. Horizontal axis corresponds to distinct pairs of alpha-phi and vertical axis is neighborhood index. Each curve corresponds to a different value of complex contagion threshold. Red curve corresponds to the largest value of complex contagion threshold which has the largest value of neighborhood index while black curve corresponds to the smallest value of complex contagion threshold and it has the smallest neighborhood index.

\subsection{Phase 2}

Phase 2:
In this stage of our study, the influence of idea distribution on network structure is assessed. To do so, five different features of network structures will be measured for three different idea distributions at 27 different parameter sets. 
Furthermore, in this stage, initially the same network structure which is the caveman network is used for each simulation. 
The three different idea distributions are:
	�	Random idea distribution: To assign idea to agents regardless of the network structure
	�	Parallel idea distribution: To assign the same idea to the members of the same cluster in the Caveman network.
	�	Anti-parallel idea distribution: To assign distinct idea to different members of the same cluster in the Caveman network.
And five different features of the network structure are:
	�	Clustering coefficient
	�	Degree distribution
	�	Number of connected components
	�	Average path length
	�	Diameter of the network
For 27 combinations of the following three parameters:
	�	Alpha: The probability that a random person in the network comes up with a novel idea at each time step of the simulation.
	�	Phi: The probability that a randomly chosen person deletes its connection with a holder of an alternative idea and makes a new connection with a holder of the same idea. The complement of phi (1-phi) is the probability that a randomly chosen person changes its idea to that of one of its neighbors.
	�	Complex contagion threshold: In order for a randomly chosen person to change its idea to that of one of its neighbors, the fraction of the neighbors holding that idea should exceed a threshold called complex contagion threshold.  






Results:
Surprisingly, we observed that the influence of parallel idea distribution on the features of the network structure is dramatically different from those of random idea distribution and anti-parallel idea distribution (Figure 9). As we can see the network diameter, average shortest path and clustering coefficient of the network influenced by parallel idea distribution are dramatically higher than those of networks influence by antiparallel and random idea distribution. However, the number of connected components of all the final networks is the same and equals one, so all agents in the networks are reachable from each other.
        
Figure 9: The influence of idea distribution on average path length (upper left), network diameter (upper right), clustering coefficient (bottom left) and number of connected components (bottom right) for 27 distinct sets of parameters (triplets of alpha-phi-threshold). 

Furthermore the degree distribution of the final network influenced by parallel idea distribution was different from those influenced by random and antiparallel idea distribution (Figure 10). The degree distribution of the network with agents by parallel idea distribution was similar to a scale free graph while the degree distribution of the networks with agents by random and antiparallel idea distribution was similar to random graph.

Figure 10: The influence of idea distribution on the degree distribution of the network. Horizontal axis corresponds to the degree of the nodes and vertical axis corresponds to the relative frequency of the nodes with that certain degree.                 
   Next, we studied the dependence of the features of network structures on parameters. In each study we keep two parameters constant and one is variable. For instance when we want to study the influence of alpha on a feature, for each of the three values of alpha, we�ll measure the feature for 3?3=9 distinct pairs of the other two parameters, so we�ll obtain 3 vectors of length 9 for each study. 

Dependence on alpha:
Unlike the influence of network structure on idea which was strongly alpha dependent, here the features of network structure are not influenced by alpha, so the influence of idea distribution on network structure is independent of creativity of scientific society (Figures 11, 12 and 13).
 
Figure 11: Dependence of clustering coefficient on alpha for each idea distribution. Horizontal axis corresponds to distinct pairs of phi-threshold and   vertical axis is the clustering coefficient. Each curve corresponds to a different value of alpha. There is no dependence on alpha for clustering coefficient of the networks with any idea distribution.

  
Figure 12: Dependence of network diameter on alpha for each idea distribution. Horizontal axis corresponds to distinct pairs of phi-threshold and vertical axis is the network diameter. Each curve corresponds to a different value of alpha. There is no dependence on alpha for the diameters of the networks with any idea distribution.

 
Figure 13: Dependence of average path length on alpha for each idea distribution. Horizontal axis corresponds to distinct pairs of phi-threshold and   vertical axis is the average path length. Each curve corresponds to a different value of alpha. There is no dependence on alpha for average path length of the networks with any idea distribution.

Dependence on phi:
All the three features of network structure (average path length, network diameter and clustering coefficient) are strongly dependent on phi value. For each idea distribution by increasing phi, all three features will decrease. However, the dependence on phi is more profound for antiparallel and random idea distribution than parallel idea distribution (figures 14, 15 and 16).
 

Figure 11: Dependence of clustering coefficient on phi for each idea distribution. Horizontal axis corresponds to distinct pairs of alpha-     threshold and   vertical axis is the clustering coefficient. Each curve corresponds to a different value of phi. 

 

Figure 12: Dependence of network diameter on phi for each idea distribution. Horizontal axis corresponds to distinct pairs of alpha-threshold and vertical axis is the network diameter. Each curve corresponds to a different value of phi. 





Figure 13: Dependence of average path length on phi for each idea distribution. Horizontal axis corresponds to distinct pairs of alpha-threshold and vertical axis is the average path length. Each curve corresponds to a different value of phi. 
Dependence on complex contagion threshold:
None of the idea distributions showed a significant dependence of average path length and network diameter on complex contagion threshold (figures 14 and 15). However, for parallel idea distribution, by increasing complex contagion threshold the clustering coefficient increased (figure 16). 
  
 Figure 14: Dependence of average path length on complex contagion threshold for each idea distribution. Horizontal axis corresponds to distinct pairs of alpha-phi and vertical axis is the average path length. Each curve corresponds to a different value of complex contagion threshold. There is no dependence on complex contagion for average path length with any idea distribution.


  
Figure 15: Dependence of network diameter on complex contagion threshold for each idea distribution. Horizontal axis corresponds to distinct pairs of phi-alpha and vertical axis is the network diameter. Each curve corresponds to a different value of complex contagion threshold. There is no dependence on complex contagion for network diameter with any idea distribution.

 

Figure 16: Dependence of clustering coefficient on complex contagion threshold for each idea distribution. Horizontal axis corresponds to distinct pairs of phi-alpha and   vertical axis is the clustering coefficient. Each curve corresponds to a different value of complex contagion threshold.

>>>>>>> I put here the matlab codes in this folder to be able to put them in the report, later I'll figure it out how remove them
